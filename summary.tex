\documentclass[a4paper, 12pt]{article}
\usepackage{fullpage}

\usepackage{amsmath}
\usepackage{amssymb}
\usepackage{amsfonts}
\usepackage{amsthm}
\usepackage{mathtools}
\usepackage{soul}

\input{python.tex}

\usepackage{xepersian}

\settextfont{XB Zar}

\theoremstyle{definition}
\newtheorem{solution}{پاسخ}[section]
\newtheorem{solutionsection}{مورد}[solution]
\renewcommand{\thesolution}{\arabic{solution}}

\renewcommand{\baselinestretch}{1.3}

\definecolor{OliveGreen}{rgb}{0.84,1,0.64}

\begin{document}

\textbf{حسابگری زیستی؛}

\textbf{حل مسئله $N$ وزیر به کمک الگوریتم ممتیک؛}

\textbf{کامیار میرزاوزیری؛ 610396152}

\hrulefill

\section{کروموزوم‌ها و جمعیت اولیه}

اولین مرحله در حل یک مسئله به کمک الگوریتم ژنتیک تعریف کروموزوم یا همان نمایش یک پاسخ برای مسئله است. برای مسئله
$N$
وزیر به دنبال یک چینش از وزیرها در صفحهٔ شطرنج هستیم. اولین ایده این است که یک ماتریس
$N \times N$
در نظر بگیریم و وزیرها را در این ساختار نمایش دهیم. اما با کمی تفکر به این نتیجه می‌رسیم که یک جایگشت برای نمایش کافی هست. چرا که طبق صورت مسئله در هر سطر و هر ستون دقیقا یک وزیر قرار می‌گیرد پس می‌توانیم جایگشتی به طول
$N$
از اعداد
$1$
تا
$N$
در نظر بگیریم که عدد
$i$-ام
موجود در آن نشان‌دهندهٔ سطری باشد که وزیر حاضر در ستون
$i$-ام
در آن قرار گرفته. به این ترتیب نه تنها از پیچیدگی کروموزوم‌ها کم می‌شود بلکه پاسخ‌های نامطلوب بسیاری نیز حذف می‌شوند. پس جایگشت‌ها را به عنوان کروموزوم انتخاب می‌کنیم. در پیاده‌سازی پایتون کلاس زیر را تعریف می‌کنیم.

\LTR
\begin{lstlisting}[language=Python]
class Chromosome:
    def __init__(self, data=None):
    global QUEENS

    if data is None:
        self.__data = list(range(QUEENS))
        random.shuffle(self.__data)
    else:
        self.__data = data
\end{lstlisting}
\RTL

در
\texttt{constructor}
این کلاس یک جایگشت رندوم ایجاد می‌کنیم، و جمعیت اولیه را به کمک این متد ایجاد می‌کنیم.

\section{تابع برازش}

در مرحلهٔ بعد باید تابع برازش را برای مسئله تعریف کنیم یا به عبارتی دیگر بدانیم کدام کروموزوم برازندگی بیشتری دارد. همانطور که در بخش قبل اشاره کردیم وزیرها در سطر یا ستون یکسان تهدید ندارند لذا تنها لازم است تهدیدهای قطری را بررسی کنیم. برای این منظور در هر یک از قطرهای اصلی و فرعی، چنانچه تعداد وزیرها بیش از یکی بود، تعداد وزیرهای اضافی را به عنوان نمره منفی برای کروموزوم در نظر می‌گیریم. با توجه به این محاسبه، مشخصهٔ
\texttt{cost}
را برای کلاس
\texttt{Chromosome}
تعریف می‌کنیم و آن را در
\texttt{constructor}
محاسبه می‌کنیم. همچنین عملگر کوچک‌تر را به کلاس اضافه می‌کنیم.

\LTR
\begin{lstlisting}[language=Python]
def __init__(self, data=None):
    ...

    self.__maindiagonals = {key: 0 for key in range(-QUEENS, QUEENS + 1)}
    self.__antidiagonals = {key: 0 for key in range(2 * QUEENS - 1)}
    self.cost = 0
    for i in range(QUEENS):
        self.__maindiagonals[i - self.__data[i]] += 1
        self.__antidiagonals[i + self.__data[i]] += 1
    diagonals = list(self.__maindiagonals.values()) + list(self.__antidiagonals.values())
    for diagonal in diagonals:
        if (diagonal > 0):
            self.cost += diagonal - 1

def __lt__(self, other):
    return self.cost > other.cost
\end{lstlisting}
\RTL

با توجه به تعریف بالا
\lr{\texttt{c1 < c2}}
به این معنی است که هزینهٔ
\texttt{c1}
از
\texttt{c2}
بیشتر است یا به عبارت دیگر
\texttt{c2}
برازنده‌تر است.

برای نمایش کروموزوم‌ها متد زیر را به کلاس
\texttt{Chromosome}
اضافه می‌کنیم.

\LTR
\begin{lstlisting}[language=Python]
def __str__(self):
    return self.__data.__str__() + ': ' + str(self.cost)
\end{lstlisting}
\RTL

حال برای مثال دو کروموزوم تصادفی ایجاد می‌کنیم و آن را نمایش می‌دهیم، همچنین بررسی می‌کنیم که کدام کروموزوم برازنده‌تر است.

\LTR
\begin{lstlisting}[language=Python]
QUEENS = 4
c1 = Chromosome()
c2 = Chromosome()
print(c1)
print(c2)
print(c1 < c2)
\end{lstlisting}
\RTL

حاصل زیر چاپ می‌شود.

\LTR
\begin{lstlisting}[language=Python]
[3, 0, 1, 2]: 3
[2, 0, 3, 1]: 0
True
\end{lstlisting}
\RTL

همانطور که دیده می‌شود هزینهٔ
\texttt{c2}
کمتر از
\texttt{c1}
است لذا برازنده‌تر است پس
\lr{\texttt{c1 < c2}}
درست تعبیر می‌شود.

\section{انتخاب}
برای انتخاب از ترکیب روش‌های انتخاب رتبه‌ای و رولت چندنشانه‌گر استفاده می‌کنیم. به این صورت که ابتدا جامعه را رتبه‌بندی می‌کنیم و سپس از رولت با
$N$
اشاره‌گر استفاده می‌کنیم. برای این منظور و همچنین کاربردهایی که در ادامه می‌بینیم، کلاس
\texttt{Population}
را پیاده‌سازی می‌کنیم.

\LTR
\begin{lstlisting}[language=Python]
class Population(list):
    def __init__(self):
        if type(countOrData) == int:
            self.__data = [Chromosome() for _ in range(countOrData)]
        elif type(countOrData) == list:
            self.__data = countOrData
        else:
            raise Exception()
        self.__data.sort()

    def __choose(self):
        n = len(self.__data)
        roulette = sum([[i] * (i + 1) for i in range(n)], [])
        turning = random.randint(0, n)
        roulette = roulette[turning:] + roulette[:turning]
        pointers = range(0, len(roulette), math.ceil(len(roulette) / n))

        choices = []
        for pointer in pointers:
            choices.append(self.__data[roulette[pointer]])

        return choices
\end{lstlisting}
\RTL


\section{بازترکیب}
حال که تکلیف جمعیت اولیه و تابع برازش مشخص شد باید برای
\texttt{crossover}
تصمیم بگیریم. به طور شهودی به نظر می‌رسد که بازترکیب ترتیبی یا
\lr{\texttt{Order Recombination}}
راه مناسبی برای ترکیب دو کروموزوم باشد. برای پیاده‌سازی بازترکیب از عملگر ضرب بین دو کروموزوم استفاده می‌کنیم.

\LTR
\begin{lstlisting}[language=Python]
def __mul__(self, other):
    global QUEENS

    (side1, side2) = random.sample(range(QUEENS + 1), 2)
    start = min(side1, side2)
    end = max(side1, side2)
    if PRINT_SLICE_INFO:
        print(start, end)
    first_child = Chromosome(self.__crossover(self.__data, other.__data, start, end))
    second_child = Chromosome(self.__crossover(other.__data, self.__data, start, end))
    return [first_child, second_child]

@staticmethod
def __crossover(mother_data: list, father_data: list, start: int, end: int):
    dimension = len(mother_data)
    data = [None] * dimension
    data[start:end] = mother_data[start:end]
    i = end
    for v in father_data[end:] + father_data[:end]:
        if v not in data:
            if i == start:
                i = end
            if i == dimension:
                i = 0
            data[i] = v
            i += 1
    return data
\end{lstlisting}
\RTL

حال برای مثال بین دو کروموزوم تصادفی عمل ضرب انجام می‌دهیم و دو فرزند به وجود آمده را بررسی می‌کنیم.

\LTR
\begin{lstlisting}[language=Python]
QUEENS = 8
c1 = Chromosome()
c2 = Chromosome()
print(c1)
print(c2)
(c3, c4) = c1 * c2
print(c3)
print(c4)
\end{lstlisting}
\RTL

حاصل زیر چاپ می‌شود.

\LTR
\begin{lstlisting}[language=Python]
[4, 7, 1, (*@\sethlcolor{OliveGreen}\hl{5, 2, 0, 3, 6}@*)]: 2
[7, 5, 0, (*@\sethlcolor{yellow}\hl{4, 2, 1, 3, 6}@*)]: 4
3 7
[7, 4, 1, (*@\sethlcolor{OliveGreen}\hl{5, 2, 0, 3, 6}@*)]: 2
[7, 5, 0, (*@\sethlcolor{yellow}\hl{4, 2, 1, 3, 6}@*)]: 4
\end{lstlisting}
\RTL

\section{جهش، جست‌وجوی محلی و الگوریتم ممتیک}
پس از تعریف
\texttt{crossover}
نوبت به تعریف جهش یا همان
\texttt{mutation}
می‌رسد. شهودا به نظر می‌رسد جهش جابه‌جایی در این مسئله بهتر از سایر جهش‌ها عمل کند لذا از این جهش استفاده می‌کنیم. تابعی پیاده‌سازی می‌کنیم که جهش جابه‌جایی را به تعداد بار مشخصی انجام دهد.

\LTR
\begin{lstlisting}[language=Python]
def swap(self, count, should_be_better):
    global QUEENS

    result = Chromosome(self.__data)

    for _ in range(count):
        (q1, q2) = random.sample(range(QUEENS), 2)
        if PRINT_SLICE_INFO:
            print(q1, q2)
        new_cost = result.cost
        new_maindiagonals = result.__maindiagonals.copy()
        new_antidiagonals = result.__antidiagonals.copy()

        new_maindiagonals[q1 - result.__data[q1]] -= 1
        if (new_maindiagonals[q1 - result.__data[q1]] >= 1):
            new_cost -= 1
        new_maindiagonals[q2 - result.__data[q2]] -= 1
        if (new_maindiagonals[q2 - result.__data[q2]] >= 1):
            new_cost -= 1
        new_antidiagonals[q1 + result.__data[q1]] -= 1
        if (new_antidiagonals[q1 + result.__data[q1]] >= 1):
            new_cost -= 1
        new_antidiagonals[q2 + result.__data[q2]] -= 1
        if (new_antidiagonals[q2 + result.__data[q2]] >= 1):
            new_cost -= 1
        new_maindiagonals[q1 - result.__data[q2]] += 1
        if (new_maindiagonals[q1 - result.__data[q2]] > 1):
            new_cost += 1
        new_maindiagonals[q2 - result.__data[q1]] += 1
        if (new_maindiagonals[q2 - result.__data[q1]] > 1):
            new_cost += 1
        new_antidiagonals[q1 + result.__data[q2]] += 1
        if (new_antidiagonals[q1 + result.__data[q2]] > 1):
            new_cost += 1
        new_antidiagonals[q2 + result.__data[q1]] += 1
        if (new_antidiagonals[q2 + result.__data[q1]] > 1):
            new_cost += 1

        if new_cost <= result.cost or not should_be_better:
            result.__data[q1], result.__data[q2] = result.__data[q2], result.__data[q1]
            result.__maindiagonals = new_maindiagonals
            result.__antidiagonals = new_antidiagonals
            result.cost = new_cost

    return result
\end{lstlisting}
\RTL

همچنین همانطور که دیده می‌شود پارامتر دیگری تحت عنوان
\texttt{should\_be\_better}
رای این تابع در نظر گرفتیم. در صورت فعال بودن این پارامتر، تابع بالا جابه‌جایی را انجام می‌دهد و دوباره هزینه را محاسبه می‌کند و تنها در صورتی که هزینه کاهش پیدا کرده باشد جابه‌جایی را اعمال می‌کند.

برای مثال یک کروموزوم تصادفی ایجاد می‌کنیم و عمل
\lr{\texttt{\_\_swap(1, True)}}
را روی آن انجام می‌دهیم. سپس نتیجه را بررسی می‌کنیم.

\LTR
\begin{lstlisting}[language=Python]
QUEENS = 8
c = Chromosome()
print(c)
print(c.__swap(1, True))
\end{lstlisting}
\RTL

حاصل زیر چاپ می‌شود.

\LTR
\begin{lstlisting}[language=Python]
[3, 1, 2, 5, 7, 0, 6, 4]: 4
2 7
[3, 1, 4, 5, 7, 0, 6, 2]: 4
\end{lstlisting}
\RTL

همانطور که می بینیم انجام جابه‌جایی هزینه را بیشتر می‌کرده و لذا انجام نشده.

در انجام جهش بدون توجه به این موضوع عمل می‌کنیم اما از همین متد در جست‌وجوی محلی نیز استفاده می‌کنیم و این پارامتر را فعال می‌کنیم و به این صورت الگوریتم ژنتیک تبدیل به الگوریتم ممتیک می‌شود. برای پیاده‌سازی جهش از عملگر معکوس یعنی
\texttt{\textasciitilde}
و برای پیاده‌سازی جست‌وجوی محلی حول یک کروموزوم از عملگر مثبت یعنی
\texttt{+}
استفاده می‌کنیم.

\LTR
\begin{lstlisting}[language=Python]
def __invert__(self):
    return self.__swap(random.randint(0, MUTATION_DEGREE), False)

def __pos__(self):
    return self.__swap(random.randint(LOCAL_SEARCH_DEGREE[0], LOCAL_SEARCH_DEGREE[1]), True)
\end{lstlisting}
\RTL

\section{جایگزینی}
به کمک کدهایی که در بالا نوشتیم می‌توانیم نسل بعدی را از روی نسل قبلی ایجاد کنیم. حال نیاز داریم تصمیم بگیریم که کدام کروموزوم‌ها را نگه داریم و کدام‌ها را حذف کنیم. تصمیم می‌گیریم
$r_0$
درصد برتر از فرزندان نسل جدید،
$r_1$
درصد از سایر فرزندان نسل جدید،
$r_2$
درصد از سایر جامعه قبلی و
$100 - r_0 - r_1 - r_2$
درصد برتر از جامعه قبلی را به عنوان جمعیت جدید در نظر بگیریم.

\LTR
\begin{lstlisting}[language=Python]
def __replacement(self, children):
    n = len(children.__data)
    best_children_count = math.floor(REPLACEMENT[0] * n)
    other_children_count = math.floor(REPLACEMENT[1] * n)
    other_parents_count = math.floor(REPLACEMENT[2] * n)
    best_parents_count = n - best_children_count - other_children_count - other_parents_count
    self.__data = (
        children.__data[-best_children_count:] +
        random.sample(children.__data[:(n - best_children_count)], other_children_count) +
        random.sample(self.__data[:(n - best_parents_count)], other_parents_count) +
        self.__data[-best_parents_count:]
    )
    self.__data.sort()
\end{lstlisting}
\RTL

\section{شرط توقف}
واضح است که توقف زمانی باید اتفاق بیافتد که هزینه برابر صفر شود، یا به عبارت دیگر هیچ دو وزیری یکدیگر را تهدید نکنند و مسئله حل شده باشد. پس متد زیر را به کلاس
\texttt{Chromosome}
اضافه می‌کنیم.

\LTR
\begin{lstlisting}[language=Python]
def solved(self):
    return self.cost == 0
\end{lstlisting}
\RTL

\section{کد نهایی}
در نهایت کد زیر را برای حل سؤال می‌نویسیم.

\LTR
\begin{lstlisting}[language=Python]
QUEENS = 1000
N = 40
MUTATION_DEGREE = 1
LOCAL_SEARCH_DEGREE = [150, 200]
REPLACEMENT = [.7, .05, .1]
ESCAPE_THRESHOLD_PROPORTION = .3
ESCAPE_PROPORTION = .5
population = Population(N)
i = 0
while True:
    if PRINT_ITERATION_NO:
        print(f"Iteration: {i}")
    if PRINT_ITERATION_BEST_ANSWER:
        print(f"Best Answer: {population.answer().cost}")
    if PRINT_ITERATION_BEST_ANSWER_DETAILS:
        print(population.answer())
    if PRINT_ITERATION_ALL_ANSWERS:
        print(f"All Answers: {population.answers()}")

    population.iterate()

    if population.answer().solved():
        break
    i += 1

print(population.answer())
\end{lstlisting}
\RTL

\section{نمایش}

کد ساده زیر را برای نمایش بصری پاسخ‌ها می‌نویسیم.

\LTR
\begin{lstlisting}[language=Python]
import sys
import json
import matplotlib.pyplot as plt
import os

BLANK = (190, 219, 57)
QUEEN = (51, 105, 30)

if (len(sys.argv) < 2):
    raise Exception('please provide a file')
path = sys.argv[1]
try:
    with open(path) as f:
        c = json.load(f)
except:
    raise Exception('invalid file')

N = len(c)
m = [[BLANK for _ in range(N)] for _ in range(N)]

for i in range(N):
    m[i][c[i]] = QUEEN


plt.imshow(m)
plt.axis('off')
plt.savefig(os.path.splitext(path)[0] + '.png')    
\end{lstlisting}
\RTL

\section{نتایج}

حال می‌توانیم کد موجود را با تعداد وزیرها متفاوت اجرا کنیم و نتایج را بررسی کنیم.

\subsection{\lr{N = 8}}

اجرای الگوریتم پس از
\texttt{1}
نسل و گذشت
\lr{\texttt{0.05}}
ثانیه متوقف شد.

\LTR
\begin{lstlisting}[language=Python]
Answer: output.8.txt
The whole process took 0.05971717834472656
\end{lstlisting}
\RTL

\begin{center}
    \includegraphics[width=.7\textwidth]{output.8.png}
\end{center}

\subsection{\lr{N = 100}}

اجرای الگوریتم پس از
\texttt{27}
نسل و گذشت
\texttt{2}
ثانیه متوقف شد.

\LTR
\begin{lstlisting}[language=Python]
Answer: output.100.txt
The whole process took 2.872501850128174
\end{lstlisting}
\RTL

\begin{center}
    \includegraphics[width=.7\textwidth]{output.100.png}
\end{center}

\subsection{\lr{N = 300}}

اجرای الگوریتم پس از
\texttt{78}
نسل و گذشت
\texttt{13}
ثانیه متوقف شد.

\LTR
\begin{lstlisting}[language=Python]
Answer: output.300.txt
The whole process took 13.421924352645874
\end{lstlisting}
\RTL

\begin{center}
    \includegraphics[width=.7\textwidth]{output.300.png}
\end{center}

\subsection{\lr{N = 1000}}

اجرای الگوریتم پس از
\texttt{412}
نسل و گذشت
\texttt{3}
دقیقه و
\texttt{28}
ثانیه متوقف شد.

\LTR
\begin{lstlisting}[language=Python]
Answer: output.1000.txt
The whole process took 208.7185254096985    
\end{lstlisting}
\RTL

\begin{center}
    \includegraphics[width=.7\textwidth]{output.1000.png}
\end{center}

\subsection{\lr{N = 2000}}

اجرای الگوریتم پس از
\texttt{970}
نسل و گذشت
\texttt{20}
دقیقه و
\texttt{54}
ثانیه متوقف شد.

\LTR
\begin{lstlisting}[language=Python]
Answer: output.2000.txt
The whole process took 1254.3887646198273
\end{lstlisting}
\RTL

\begin{center}
    \includegraphics[width=.7\textwidth]{output.2000.png}
\end{center}

\subsection{\lr{N = 5000}}

با توجه به حجم مسئله جمعیت را به ناچار کوچک می‌کنیم و پارامتر
\lr{\texttt{N = 10}}
قرار می‌دهیم.

اجرای الگوریتم پس از
\texttt{2298}
نسل و گذشت
\texttt{46}
دقیقه و
\texttt{14}
ثانیه متوقف شد.

\LTR
\begin{lstlisting}[language=Python]
Answer: output.5000.txt
The whole process took 2774.877559185028
\end{lstlisting}
\RTL

\begin{center}
    \includegraphics[width=.7\textwidth]{output.5000.png}
\end{center}

\end{document}


\LTR
\begin{lstlisting}[language=Python]
\end{lstlisting}
\RTL
